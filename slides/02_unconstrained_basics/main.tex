\documentclass[aspectratio=1610]{beamer}
\usepackage[utf8]{inputenc}
\usepackage[default,oldstyle,scale=0.95]{opensans}
\usepackage[T1]{fontenc}

\usepackage[natbib=true,style=numeric,sorting=none]{biblatex}
\addbibresource{references.bib}

\usepackage{pgf-pie}  
\usepackage{caption}
\usepackage{subcaption}
\usepackage{algorithm}
\usepackage{algpseudocode}
\usepackage{amsmath,amssymb,amsfonts}
\usepackage{svg}

\usetikzlibrary{arrows.meta}
\usetikzlibrary{tikzmark}
\usetikzlibrary{math}
\usetikzlibrary{backgrounds,calc,positioning}
\usepackage{qrcode}

\usetheme{main}

\captionsetup{font=scriptsize,labelfont=scriptsize}


\title[Numerical optimization]{Numerical optimization : theory and applications}

\date[]{}
\author[AM]{\textbf{Ammar Mian} \\ \footnotesize Associate professor, LISTIC, Université Savoie Mont Blanc}

\newcommand{\red}[1]{\textcolor{red}{#1}}
% \newcommand{\alert}[1]{{\textbf{\alert{#1}}}}


%\input{macros.tex}


\AtBeginBibliography{\scriptsize}

%%%%%%%%%%%%%%%%%%%%%%%%%%%%%%%%%%%%%%%%%%%%%%%%%%%%%%%%%%%%
%                     END OF PREAMBLE
%
%%%%%%%%%%%%%%%%%%%%%%%%%%%%%%%%%%%%%%%%%%%%%%%%%%%%%%%%%%%%

\begin{document}
%%%%%%%%%%%%%%%


  
\begin{frame}[noframenumbering,plain]
\titlepage
\end{frame}
\begingroup
\setbeamercolor{background canvas}{bg=main}
\setbeamercolor{titlelike}{fg=text-light, bg=light}
\begin{frame}[noframenumbering,plain]{Outline}
    \tableofcontents[]
\end{frame}

\endgroup

\AtBeginSection[]{
        \setbeamercolor{background canvas}{bg=main} 
    \begin{frame}[plain, noframenumbering]
        \tableofcontents[currentsection]
    \end{frame}
    \setbeamercolor{background canvas}{bg=light} 
}


\AtBeginSubsection[]
{
    \setbeamercolor{background canvas}{bg=main} 
    \begin{frame}[noframenumbering, plain, label=]
        % \frametitle{Plan}  
        \tableofcontents[currentsection,currentsubsection]
    \end{frame}
    \setbeamercolor{background canvas}{bg=light} 
}



%%%%%%%%%%%%%%%%%%%%%%%%%%%%%%%%%%%%%%%%%%%%%%%%%%%%%%%%%%%%
%%%%%%%%%%%%%%%%%%%%%%%%%%%%%%%%%%%%%%%%%%%%%%%%%%%%%%%%%%%%


\section{Unconstrained Optimization - Basics}

\begin{frame}{Problem Formulation}
  \begin{block}{Unconstrained Optimization Problem}
    We aim to solve:
    $$\underset{\mathbf{x}\in\mathbb{R}^d}{\operatorname{argmin}} f(\mathbf{x})$$
    
    where:
    \begin{itemize}
      \item $\mathbf{x} \in \mathbb{R}^d$ is the optimization variable
      \item $f: \mathcal{D}_f \mapsto \mathbb{R}$ is the objective function
      \item No constraints on the admissible solutions
    \end{itemize}
  \end{block}
  
  \vspace{0.5cm}
  \textbf{Goal:} Characterize the nature of solutions under this setup.
\end{frame}

\section{What is a Solution?}

\begin{frame}{Local vs Global Minima}
  % Empty frame for Figure: Local and global minimum can coexist
  \begin{figure}[h]
    \centering
    \includegraphics[height=0.8\textheight]{figures/local_global_minima.pdf}
    \caption{Local and global minima can coexist in the same function.}
  \end{figure}
\end{frame}

\begin{frame}{Global Minimizer}
  \begin{definition}[Global minimizer]
    A point $\mathbf{x}^\star$ is a \textbf{global minimizer} if 
    $$f(\mathbf{x}^\star) \leq f(\mathbf{x})$$
    where $\mathbf{x}$ ranges over all of $\mathbb{R}^d$ (or at least over the domain of interest).
  \end{definition}
  
  \vspace{0.5cm}
  \begin{itemize}
    \item Global minimizers can be difficult to find
    \item Our knowledge of $f$ is usually only local
    \item Algorithms typically don't visit many points
    \item Cannot guarantee finding global minimum in general
  \end{itemize}
\end{frame}

\begin{frame}{Local Minimizer}
  \begin{definition}[Local minimizer]
    A point $\mathbf{x}^\star$ is a \textbf{local minimizer} if 
    $$\exists r > 0, \quad f(\mathbf{x}^\star) \leq f(\mathbf{x}), \quad \forall \mathbf{x} \in \mathcal{B}(\mathbf{x}^\star, r)$$
  \end{definition}
  
  \vspace{0.5cm}
  \begin{block}{Types of Local Minimizers}
    \begin{itemize}
      \item \textbf{Weak local minimizer:} satisfies the definition above
      \item \textbf{Strict local minimizer:} when $f(\mathbf{x}^\star) < f(\mathbf{x})$ for all $\mathbf{x} \neq \mathbf{x}^\star$ in the neighborhood
    \end{itemize}
  \end{block}
\end{frame}

\section{Taylor's Theorem and Optimality Conditions}

\begin{frame}{Taylor's Theorem}
  \begin{theorem}[Taylor's theorem]
    Suppose $f:\mathbb{R}^d \mapsto \mathbb{R}$ is continuously differentiable and $\mathbf{p} \in \mathbb{R}^d$. Then:
    $$f(\mathbf{x}+\mathbf{p}) = f(\mathbf{x}) + \nabla f(\mathbf{x}+t\mathbf{p})^T\mathbf{p}$$
    for some $t \in [0,1]$.
    
    \vspace{0.3cm}
    Moreover, if $f$ is twice continuously differentiable:
    $$f(\mathbf{x}+\mathbf{p}) = f(\mathbf{x}) + \nabla f(\mathbf{x})^T\mathbf{p} + \frac{1}{2}\mathbf{p}^T\nabla^2 f(\mathbf{x}+t\mathbf{p})\mathbf{p}$$
    for some $t \in [0,1]$.
  \end{theorem}
\end{frame}

\begin{frame}{Taylor's Approximation}
  \begin{theorem}[Taylor's approximation]
    \textbf{First order approximation:}
    $$f(\mathbf{x}+\mathbf{p}) = f(\mathbf{x}) + \nabla f(\mathbf{x})^T\mathbf{p} + o(\|\mathbf{p}\|)$$
    
    \vspace{0.3cm}
    \textbf{Second-order approximation:}
    $$f(\mathbf{x}+\mathbf{p}) = f(\mathbf{x}) + \nabla f(\mathbf{x})^T\mathbf{p} + \frac{1}{2}\mathbf{p}^T\nabla^2 f(\mathbf{x})\mathbf{p} + o(\|\mathbf{p}\|^2)$$
    
    where $o(\|\mathbf{p}\|)$ and $o(\|\mathbf{p}\|^2)$ represent terms that grow slower than $\|\mathbf{p}\|$ and $\|\mathbf{p}\|^2$ respectively as $\|\mathbf{p}\| \to 0$.
  \end{theorem}
\end{frame}

\begin{frame}{First-Order Necessary Conditions}
  \begin{theorem}[First-order necessary conditions]
    If $\mathbf{x}^\star$ is a local minimizer, and $f$ is continuously differentiable in a neighborhood of $\mathbf{x}^\star$, then 
    $$\nabla f(\mathbf{x}^\star) = \mathbf{0}$$
  \end{theorem}
  
  \vspace{0.5cm}
  \begin{block}{Stationary Points}
    We call any point $\mathbf{x}$ such that $\nabla f(\mathbf{x}) = \mathbf{0}$ a \textbf{stationary point}.
  \end{block}
\end{frame}

\begin{frame}{Matrix Definiteness}
  \begin{block}{Definitions}
    A matrix $\mathbf{B}$ is:
    \begin{itemize}
      \item \textbf{Positive definite} if $\mathbf{p}^T \mathbf{B} \mathbf{p} > 0$ for all $\mathbf{p} \neq \mathbf{0}$
      \item \textbf{Positive semidefinite} if $\mathbf{p}^T \mathbf{B} \mathbf{p} \geq 0$ for all $\mathbf{p}$
    \end{itemize}
  \end{block}
\end{frame}

\begin{frame}{Second-Order Necessary Conditions}
  \begin{theorem}[Second-order necessary conditions]
    If $\mathbf{x}^\star$ is a local minimizer of $f$ and $\nabla^2 f$ is continuous in an open neighborhood of $\mathbf{x}^\star$, then:
    \begin{itemize}
      \item $\nabla f(\mathbf{x}^\star) = \mathbf{0}$
      \item $\nabla^2 f(\mathbf{x}^\star)$ is positive semidefinite
    \end{itemize}
  \end{theorem}
\end{frame}

\begin{frame}{Second-Order Sufficient Conditions}
  \begin{theorem}[Second-Order Sufficient Conditions]
    Suppose that $\nabla^2 f$ is continuous in an open neighborhood of $\mathbf{x}^\star$ and that:
    \begin{itemize}
      \item $\nabla f(\mathbf{x}^\star) = \mathbf{0}$
      \item $\nabla^2 f(\mathbf{x}^\star)$ is positive definite
    \end{itemize}
    Then $\mathbf{x}^\star$ is a strict local minimizer of $f$.
  \end{theorem}
  
  \vspace{0.3cm}
  \begin{block}{Note}
    These sufficient conditions are not necessary. Example: $f(x) = x^4$ at $x^\star = 0$ is a strict local minimizer but the Hessian vanishes.
  \end{block}
\end{frame}

\section{Optimization Algorithms}

\begin{frame}{The Need for Algorithms}
  \begin{block}{Why do we need algorithms?}
    \begin{itemize}
      \item We know that $\nabla f(\mathbf{x}^\star) = \mathbf{0}$ characterizes local minima
      \item But we don't always have the luxury to solve $\nabla f(\mathbf{x}) = \mathbf{0}$ analytically
      \item Computing and checking Hessian conditions can be expensive
    \end{itemize}
  \end{block}
  
  \vspace{0.5cm}
  \begin{block}{Algorithmic Approach}
    Design iterative algorithms that update $\mathbf{x}$ until convergence to a local minimizer:
    \begin{itemize}
      \item \textbf{Gradient-based methods:} use only gradient information
      \item \textbf{Newton methods:} use Hessian to accelerate convergence
      \item \textbf{Quasi-Newton methods:} approximate Hessian for balance
    \end{itemize}
  \end{block}
\end{frame}

\subsection{Steepest Descent Method}

\begin{frame}{Steepest Descent Algorithm}
  \begin{block}{Algorithm}
    Choose initial point $\mathbf{x}_0$ and compute:
    $$\mathbf{x}_{k+1} = \mathbf{x}_k - \alpha_k \nabla f(\mathbf{x}_k)$$
    where $\alpha_k$ are scalar values called \textbf{step-size} (or learning rate).
  \end{block}
  
  \vspace{0.5cm}
  \begin{block}{Intuition}
    \begin{itemize}
      \item Like walking down a mountain in fog
      \item Feel the slope and step in steepest descent direction
      \item $-\nabla f(\mathbf{x}_k)$ points in direction of steepest decrease
      \item Most aggressive local progress toward reducing function value
    \end{itemize}
  \end{block}
\end{frame}

\begin{frame}{Steepest Descent - Successful Optimization}
  % Empty frame for Figure: Optimization with steepest descent
    \begin{figure}[h]
    \centering
      \includegraphics[height=0.8\textheight]{figures/steepest_successful.pdf}
    \caption{Successful optimization with steepest descent.}
  \end{figure}

\end{frame}

\begin{frame}{Step Size Trade-offs}
  \begin{block}{Step Size $\alpha_k$ Considerations}
    The choice of step size involves a fundamental trade-off:
    \begin{itemize}
      \item \textbf{Too small:} Painfully slow progress
      \item \textbf{Too large:} Might overshoot or start climbing uphill
    \end{itemize}
  \end{block}
\end{frame}

\begin{frame}{Steepest Descent - Zigzag Problem}
  % Empty frame for Figure: Problem of stepsize
    \begin{figure}[h]
    \centering
      \includegraphics[height=0.8\textheight]{figures/steepest_zigzag.pdf}
    \caption{Zigzag behavior of steepest descent in narrow valleys.}
  \end{figure}

\end{frame}

\begin{frame}{Why Steepest Descent Can Struggle}
  \begin{block}{Zigzag Behavior}
    \begin{itemize}
      \item Occurs in narrow valley-like functions (large condition number)
      \item Algorithm bounces between valley walls instead of walking down
      \item Steepest direction points toward walls, not down the valley
      \item Fundamentally myopic: only considers immediate local slope
    \end{itemize}
  \end{block}
  
  \vspace{0.5cm}
  \begin{block}{Convergence Properties}
    \begin{itemize}
      \item Linear convergence under reasonable conditions
      \item Error decreases by constant factor each iteration
      \item Can be frustratingly slow for poorly conditioned problems
    \end{itemize}
  \end{block}
\end{frame}

\subsection{Newton Method}

\begin{frame}{Newton Method - Motivation}
  \begin{block}{Key Insight}
    \begin{itemize}
      \item Steepest descent: navigating with only immediate slope
      \item Newton method: having detailed topographic map
      \item Incorporates curvature information (how slope changes)
      \item Uses second-order Taylor approximation
    \end{itemize}
  \end{block}
  
  \vspace{0.5cm}
  \begin{block}{Strategy}
    Instead of minimizing $f$ directly, minimize simpler quadratic approximation:
    $$f(\mathbf{x}_k + \mathbf{p}) \approx f(\mathbf{x}_k) + \nabla f(\mathbf{x}_k)^T\mathbf{p} + \frac{1}{2}\mathbf{p}^T\nabla^2 f(\mathbf{x}_k)\mathbf{p}$$
  \end{block}
\end{frame}

\begin{frame}{Newton Method - Algorithm}
  \begin{block}{Derivation}
    Setting gradient of quadratic approximation to zero:
    $$\nabla f(\mathbf{x}_k) + \nabla^2 f(\mathbf{x}_k)\mathbf{p} = \mathbf{0}$$
    
    Solving for Newton step:
    $$\mathbf{p}_k = -[\nabla^2 f(\mathbf{x}_k)]^{-1}\nabla f(\mathbf{x}_k)$$
  \end{block}
  
  \vspace{0.5cm}
  \begin{block}{Newton Iteration}
    $$\mathbf{x}_{k+1} = \mathbf{x}_k - [\nabla^2 f(\mathbf{x}_k)]^{-1}\nabla f(\mathbf{x}_k)$$
  \end{block}
\end{frame}

\begin{frame}{Newton Method - Optimization Step}
  % Empty frame for Figure: Newton optimization step
    \begin{figure}[h]
    \centering
      \includegraphics[height=0.8\textheight]{figures/newton_step.pdf}
    \caption{Newton method optimization step.}
  \end{figure}
  %
\end{frame}

\begin{frame}{Newton Method - Geometric Insight}
  \begin{block}{Hessian Information}
    \begin{itemize}
      \item $\nabla^2 f(\mathbf{x}_k)$ encodes how gradient changes in different directions
      \item Recognizes elongated valley shapes
      \item Takes larger steps along valley floor, smaller steps perpendicular
      \item Eliminates zigzag behavior of steepest descent
    \end{itemize}
  \end{block}
  
  \vspace{0.5cm}
  \begin{block}{Special Property}
    For quadratic functions: Newton method finds exact minimum in single step, regardless of conditioning!
  \end{block}
\end{frame}

\begin{frame}{Newton Method - Convergence}
  \begin{block}{Quadratic Convergence}
    Near solution satisfying second-order sufficient conditions:
    \begin{itemize}
      \item Number of correct digits roughly doubles each iteration
      \item If 1 correct digit → next iteration gives 2 → then 4 → then 8
      \item Incredibly efficient for high-precision optimization
      \item Forms backbone of many sophisticated algorithms
    \end{itemize}
  \end{block}
  
  \vspace{0.5cm}
  \begin{block}{Comparison}
    \begin{itemize}
      \item \textbf{Linear convergence:} 1 digit → 3 iterations → 2 digits
      \item \textbf{Quadratic convergence:} 1 digit → 1 iteration → 2 digits
    \end{itemize}
  \end{block}
\end{frame}

\begin{frame}{Newton Method - Computational Cost}
  \begin{block}{The Price of Power}
    \begin{itemize}
      \item Must compute Hessian matrix: $d(d+1)/2$ second derivatives
      \item Must solve linear system: $\nabla^2 f(\mathbf{x}_k)\mathbf{p}_k = -\nabla f(\mathbf{x}_k)$
      \item Requires $\sim d^3/3$ arithmetic operations per iteration
      \item Becomes prohibitive as dimension $d$ grows
    \end{itemize}
  \end{block}
\end{frame}

\begin{frame}{Newton Method - Potential Failures}
  \begin{block}{When Newton's Method Can Fail}
    \begin{itemize}
      \item Hessian might not be positive definite away from minimum
      \item Quadratic model might have maximum or saddle point
      \item Newton step might point in wrong direction
      \item Poor quadratic approximation far from minimum
      \item Steps might increase function value
    \end{itemize}
  \end{block}
\end{frame}

\section{Looking Ahead}

\begin{frame}{Bridge Between Methods}
  \begin{block}{Motivating Questions}
    \begin{itemize}
      \item Can we capture Newton's geometric insight without full computational burden?
      \item Can we ensure global reliability while achieving faster convergence?
    \end{itemize}
  \end{block}
  
  \vspace{0.5cm}
  \begin{block}{Advanced Methods}
    \begin{itemize}
      \item \textbf{Quasi-Newton methods (BFGS):} Approximate Hessian using gradients, superlinear convergence
      \item \textbf{Trust region methods:} Systematic progress guarantees
      \item \textbf{Line search strategies:} Reliable step size selection
    \end{itemize}
  \end{block}
\end{frame}






\end{document}


