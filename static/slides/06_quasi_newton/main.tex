\documentclass[aspectratio=1610]{beamer}
\usepackage[utf8]{inputenc}
% \usepackage[default,oldstyle,scale=0.95]{opensans}
% \usepackage[T1]{fontenc}
\usepackage{sansmathfonts}
\usepackage[T1]{fontenc}

\usepackage[natbib=true,style=numeric,sorting=none]{biblatex}
\addbibresource{references.bib}

\usepackage{tikz}
\usepackage{pgf-pie}  
\usepackage{caption}
\usepackage{subcaption}
\usepackage{algorithm}
\usepackage{algpseudocode}
\usepackage{amsmath,amssymb,amsfonts}
\usepackage{svg}

\usetikzlibrary{arrows.meta}
\usetikzlibrary{tikzmark}
\usetikzlibrary{math}
\usetikzlibrary{backgrounds,calc,positioning}
\usepackage{qrcode}
\usepackage{siunitx}
\usetikzlibrary{patterns,positioning,arrows.meta,decorations.pathreplacing}

\usepackage{pgfplots}
\usepackage{algorithm, algpseudocode}



\usetheme{main}

\captionsetup{font=scriptsize,labelfont=scriptsize}


\title[Numerical optimization]{Numerical optimization : theory and applications}

\date[]{}
\author[AM]{\textbf{Ammar Mian} \\ \footnotesize Associate professor, LISTIC, Université Savoie Mont Blanc}

\newcommand{\red}[1]{\textcolor{red}{#1}}
% \newcommand{\alert}[1]{{\textbf{\alert{#1}}}}


%\input{macros.tex}


\AtBeginBibliography{\scriptsize}

%%%%%%%%%%%%%%%%%%%%%%%%%%%%%%%%%%%%%%%%%%%%%%%%%%%%%%%%%%%%
%                     END OF PREAMBLE
%
%%%%%%%%%%%%%%%%%%%%%%%%%%%%%%%%%%%%%%%%%%%%%%%%%%%%%%%%%%%%

\begin{document}
%%%%%%%%%%%%%%%


  
\begin{frame}[noframenumbering,plain]
\titlepage
\end{frame}
\begingroup
\setbeamercolor{background canvas}{bg=main}
\setbeamercolor{titlelike}{fg=text-light, bg=light}
\begin{frame}[noframenumbering,plain]{Outline}
    \tableofcontents[]
\end{frame}

\endgroup

\AtBeginSection[]{
        \setbeamercolor{background canvas}{bg=main} 
    \begin{frame}[plain, noframenumbering]
        \tableofcontents[currentsection]
    \end{frame}
    \setbeamercolor{background canvas}{bg=light} 
}


\AtBeginSubsection[]
{
    \setbeamercolor{background canvas}{bg=main} 
    \begin{frame}[noframenumbering, plain, label=]
        % \frametitle{Plan}  
        \tableofcontents[currentsection,currentsubsection]
    \end{frame}
    \setbeamercolor{background canvas}{bg=light} 
}



%%%%%%%%%%%%%%%%%%%%%%%%%%%%%%%%%%%%%%%%%%%%%%%%%%%%%%%%%%%%
%%%%%%%%%%%%%%%%%%%%%%%%%%%%%%%%%%%%%%%%%%%%%%%%%%%%%%%%%%%%
\subsection{Newton Method}

\begin{frame}{Newton Method - Motivation}
  \begin{block}{Key Insight}
    \begin{itemize}
      \item Steepest descent: navigating with only immediate slope
      \item Newton method: having detailed topographic map
      \item Incorporates curvature information (how slope changes)
      \item Uses second-order Taylor approximation
    \end{itemize}
  \end{block}
  
  \vspace{0.5cm}
  \begin{block}{Strategy}
    Instead of minimizing $f$ directly, minimize simpler quadratic approximation:
    $$f(\mathbf{x}_k + \mathbf{p}) \approx f(\mathbf{x}_k) + \nabla f(\mathbf{x}_k)^T\mathbf{p} + \frac{1}{2}\mathbf{p}^T\nabla^2 f(\mathbf{x}_k)\mathbf{p}$$
  \end{block}
\end{frame}

\begin{frame}{Newton Method - Algorithm}
  \begin{block}{Derivation}
    Setting gradient of quadratic approximation to zero:
    $$\nabla f(\mathbf{x}_k) + \nabla^2 f(\mathbf{x}_k)\mathbf{p} = \mathbf{0}$$
    
    Solving for Newton step:
    $$\mathbf{p}_k^N = -[\nabla^2 f(\mathbf{x}_k)]^{-1}\nabla f(\mathbf{x}_k)$$
  \end{block}
  
  \vspace{0.5cm}
  \begin{block}{Newton Iteration}
    $$\mathbf{x}_{k+1} = \mathbf{x}_k - [\nabla^2 f(\mathbf{x}_k)]^{-1}\nabla f(\mathbf{x}_k)$$
  \end{block}
\end{frame}

\begin{frame}{Newton Method - Properties}
  \begin{block}{Advantages}
    \begin{itemize}
      \item Recognizes elongated valley shapes via Hessian
      \item Takes larger steps along valley floor, smaller steps perpendicular
      \item Eliminates zigzag behavior of steepest descent
      \item Natural step size of $\alpha = 1$
      \item Quadratic convergence rate
    \end{itemize}
  \end{block}
  
  \vspace{0.5cm}
  \begin{block}{Special Property}
    For quadratic functions: Newton method finds exact minimum in single step, regardless of conditioning!
  \end{block}
\end{frame}

\begin{frame}{Newton Method - Challenges}
  \begin{block}{Main Drawbacks}
    \begin{itemize}
      \item Requires computation of Hessian matrix $\nabla^2 f(\mathbf{x})$
      \item Need to solve linear system at each iteration
      \item Hessian may not be positive definite away from solution
      \item Expensive: $O(n^3)$ operations per iteration
    \end{itemize}
  \end{block}
  
  \vspace{0.5cm}
  \begin{block}{When Newton Fails}
    When $\nabla^2 f_k$ is not positive definite:
    \begin{itemize}
      \item Newton direction may not be defined
      \item May not satisfy descent condition $\nabla f_k^T \mathbf{p}_k^N < 0$
    \end{itemize}
  \end{block}
\end{frame}

\subsection{Quasi-Newton Methods}

\begin{frame}{Quasi-Newton Methods - Motivation}
  \begin{block}{Core Idea}
    \begin{itemize}
      \item Avoid computing exact Hessian $\nabla^2 f_k$
      \item Use approximation $\mathbf{B}_k \approx \nabla^2 f_k$
      \item Update approximation using gradient information
      \item Achieve superlinear convergence without Hessian computation
    \end{itemize}
  \end{block}
  
  \vspace{0.5cm}
  \begin{block}{Quasi-Newton Direction}
    $$\mathbf{p}_k = -\mathbf{B}_k^{-1} \nabla f_k$$
    where $\mathbf{B}_k$ is updated after each step.
  \end{block}
\end{frame}

\begin{frame}{The Secant Equation}
  \begin{block}{Key Requirement}
    We want $\mathbf{B}_{k+1}$ to satisfy:
    $$\mathbf{B}_{k+1} \mathbf{s}_k = \mathbf{y}_k$$
    where:
    \begin{itemize}
      \item $\mathbf{s}_k = \mathbf{x}_{k+1} - \mathbf{x}_k$ (displacement)
      \item $\mathbf{y}_k = \nabla f_{k+1} - \nabla f_k$ (gradient change)
    \end{itemize}
  \end{block}
  
  \vspace{0.5cm}
  \begin{block}{Curvature Condition}
    For positive definite updates, we need:
    $$\mathbf{s}_k^T \mathbf{y}_k > 0$$
    This is guaranteed by Wolfe line search conditions.
  \end{block}
\end{frame}

\subsection{BFGS Method}

\begin{frame}{BFGS Method}
  \begin{block}{Most Popular Quasi-Newton Method}
    Named after Broyden, Fletcher, Goldfarb, and Shanno.
  \end{block}
  
  \vspace{0.5cm}
  \begin{block}{BFGS Update Formula}
    $$\mathbf{H}_{k+1} = \left(\mathbf{I} - \rho_k \mathbf{s}_k \mathbf{y}_k^T\right) \mathbf{H}_k \left(\mathbf{I} - \rho_k \mathbf{y}_k \mathbf{s}_k^T\right) + \rho_k \mathbf{s}_k \mathbf{s}_k^T$$
    where:
    \begin{itemize}
      \item $\mathbf{H}_k = \mathbf{B}_k^{-1}$ (inverse Hessian approximation)
      \item $\rho_k = \frac{1}{\mathbf{y}_k^T \mathbf{s}_k}$
    \end{itemize}
  \end{block}
\end{frame}

\begin{frame}{BFGS Algorithm}
  \begin{block}{Algorithm Steps}
    \begin{enumerate}
      \item Choose initial $\mathbf{x}_0$ and $\mathbf{H}_0$ (often $\mathbf{H}_0 = \mathbf{I}$)
      \item While $\|\nabla f_k\| > \epsilon$:
      \begin{itemize}
        \item Compute search direction: $\mathbf{p}_k = -\mathbf{H}_k \nabla f_k$
        \item Line search: find $\alpha_k$ satisfying Wolfe conditions
        \item Update: $\mathbf{x}_{k+1} = \mathbf{x}_k + \alpha_k \mathbf{p}_k$
        \item Compute: $\mathbf{s}_k = \mathbf{x}_{k+1} - \mathbf{x}_k$, $\mathbf{y}_k = \nabla f_{k+1} - \nabla f_k$
        \item Update $\mathbf{H}_{k+1}$ using BFGS formula
      \end{itemize}
    \end{enumerate}
  \end{block}
\end{frame}

\begin{frame}{BFGS Properties}
  \begin{block}{Key Advantages}
    \begin{itemize}
      \item Only $O(n^2)$ operations per iteration
      \item Superlinear convergence rate
      \item Maintains positive definiteness automatically
      \item Self-correcting: bad approximations get corrected
      \item No second derivatives required
    \end{itemize}
  \end{block}
  
  \vspace{0.5cm}
  \begin{block}{Convergence Comparison}
    \begin{center}
    \begin{tabular}{|l|c|c|}
    \hline
    Method & Steepest Descent & BFGS \\
    \hline
    Iterations & 5264 & 34 \\
    Convergence & Linear & Superlinear \\
    \hline
    \end{tabular}
    \end{center}
    Example on Rosenbrock function from $(-1.2, 1)$.
  \end{block}
\end{frame}

\subsection{SR1 Method}

\begin{frame}{Symmetric Rank-1 (SR1) Method}
  \begin{block}{Rank-1 Update}
    $$\mathbf{B}_{k+1} = \mathbf{B}_k + \frac{(\mathbf{y}_k - \mathbf{B}_k \mathbf{s}_k)(\mathbf{y}_k - \mathbf{B}_k \mathbf{s}_k)^T}{(\mathbf{y}_k - \mathbf{B}_k \mathbf{s}_k)^T \mathbf{s}_k}$$
  \end{block}
  
  \vspace{0.5cm}
  \begin{block}{Key Differences from BFGS}
    \begin{itemize}
      \item Rank-1 update (vs. rank-2 for BFGS)
      \item Does not maintain positive definiteness
      \item Can handle indefinite Hessians
      \item Often produces better Hessian approximations
    \end{itemize}
  \end{block}
\end{frame}

\begin{frame}{SR1 Implementation Issues}
  \begin{block}{Potential Problems}
    \begin{itemize}
      \item Denominator can vanish: $(\mathbf{y}_k - \mathbf{B}_k \mathbf{s}_k)^T \mathbf{s}_k = 0$
      \item No symmetric rank-1 update may exist
      \item Numerical instabilities possible
    \end{itemize}
  \end{block}
  
  \vspace{0.5cm}
  \begin{block}{Safeguard Strategy}
    Skip update when:
    $$|\mathbf{s}_k^T(\mathbf{y}_k - \mathbf{B}_k \mathbf{s}_k)| < r\|\mathbf{s}_k\|\|\mathbf{y}_k - \mathbf{B}_k \mathbf{s}_k\|$$
    where $r \approx 10^{-8}$ is small tolerance.
  \end{block}
\end{frame}

\begin{frame}{SR1 - Finite Termination Property}
  \begin{block}{Remarkable Property}
    For quadratic functions, SR1 method:
    \begin{itemize}
      \item Converges to minimizer in at most $n$ steps
      \item Satisfies secant equation for \textbf{all} previous directions
      \item Recovers exact Hessian: $\mathbf{H}_n = A^{-1}$ after $n$ steps
    \end{itemize}
  \end{block}
  
  \vspace{0.5cm}
  \begin{block}{Advantage over BFGS}
    This property holds regardless of line search accuracy, while BFGS requires exact line search for similar guarantees.
  \end{block}
\end{frame}

\subsection{Convergence Theory}

\begin{frame}{Global Convergence}
  \begin{block}{Zoutendijk's Condition}
    For line search methods satisfying Wolfe conditions:
    $$\sum_{k=0}^{\infty} \cos^2 \theta_k \|\nabla f_k\|^2 < \infty$$
    where $\theta_k$ is angle between search direction and negative gradient.
  \end{block}
  
  \vspace{0.5cm}
  \begin{block}{Newton-like Methods}
    If $\mathbf{p}_k = -\mathbf{B}_k^{-1} \nabla f_k$ with bounded condition number:
    $$\|\mathbf{B}_k\|\|\mathbf{B}_k^{-1}\| \leq M$$
    Then: $\cos \theta_k \geq 1/M$ and $\lim_{k \to \infty} \|\nabla f_k\| = 0$.
  \end{block}
\end{frame}

\begin{frame}{Rate of Convergence}
  \begin{block}{Convergence Rates}
    \begin{itemize}
      \item \textbf{Steepest Descent:} Linear convergence
      \item \textbf{Newton:} Quadratic convergence (near solution)
      \item \textbf{Quasi-Newton:} Superlinear convergence
    \end{itemize}
  \end{block}
  
  \vspace{0.5cm}
  \begin{block}{Practical Performance}
    \begin{itemize}
      \item Newton: Fastest per iteration, but expensive
      \item BFGS: Good balance of speed and cost
      \item Steepest Descent: Slow but simple and robust
    \end{itemize}
  \end{block}
\end{frame}

\begin{frame}{Implementation Considerations}
  \begin{block}{Step Size Strategy}
    \begin{itemize}
      \item Always try $\alpha = 1$ first (Newton step)
      \item Use Wolfe conditions for line search
      \item BFGS: accept $\alpha = 1$ eventually for superlinear convergence
    \end{itemize}
  \end{block}
  
  \vspace{0.5cm}
  \begin{block}{Initial Hessian Approximation}
    Common choices for $\mathbf{H}_0$:
    \begin{itemize}
      \item Identity matrix: $\mathbf{H}_0 = \mathbf{I}$
      \item Scaled identity: $\mathbf{H}_0 = \beta \mathbf{I}$
      \item After first step: $\mathbf{H}_0 = \frac{\mathbf{y}_0^T \mathbf{s}_0}{\mathbf{y}_0^T \mathbf{y}_0} \mathbf{I}$
    \end{itemize}
  \end{block}
\end{frame}

\begin{frame}{Summary}
  \begin{block}{Method Comparison}
    \begin{center}
    \begin{tabular}{|l|c|c|c|}
    \hline
    Method & Cost/Iter & Convergence & Hessian \\
    \hline
    Steepest Descent & $O(n)$ & Linear & Not needed \\
    Newton & $O(n^3)$ & Quadratic & Required \\
    BFGS & $O(n^2)$ & Superlinear & Approximated \\
    SR1 & $O(n^2)$ & Superlinear & Approximated \\
    \hline
    \end{tabular}
    \end{center}
  \end{block}
  
  \vspace{0.5cm}
  \begin{block}{Practical Recommendation}
    BFGS is the most widely used method due to its excellent balance of:
    \begin{itemize}
      \item Fast convergence (superlinear)
      \item Moderate computational cost
      \item Robust performance
      \item No second derivatives required
    \end{itemize}
  \end{block}
\end{frame}

\subsection{Exercises}

\begin{frame}{Exercise 1: Himmelblau Function}
  \begin{block}{Problem Statement}
    Implement BFGS and SR1 methods to minimize the Himmelblau function:
    $f(x_1, x_2) = (x_1^2 + x_2 - 11)^2 + (x_1 + x_2^2 - 7)^2$
  \end{block}
  
  \vspace{0.5cm}
  \begin{block}{Tasks}
    \begin{enumerate}
      \item Compute the gradient $\nabla f(x_1, x_2)$ analytically
      \item Implement both BFGS and SR1 algorithms with Wolfe line search
      \item Test from starting points: $(0, 0)$, $(1, 1)$, $(-1, 1)$, $(4, 4)$
      \item Compare convergence behavior, number of iterations, and final solutions
      \item Plot convergence trajectories on contour plot
    \end{enumerate}
  \end{block}
  
\end{frame}

\begin{frame}{Exercise 2: Mixed Function}
  \begin{block}{Problem Statement}
    Implement BFGS and SR1 methods to minimize:
    $f(x_1, x_2) = \frac{1}{2}x_1^2 + x_1\cos(x_2)$
  \end{block}
  
  \vspace{0.5cm}
  \begin{block}{Tasks}
    \begin{enumerate}
      \item Derive the gradient $\nabla f(x_1, x_2)$ and Hessian $\nabla^2 f(x_1, x_2)$
      \item Implement BFGS, SR1, and exact Newton method
      \item Use starting points: $(1, 0)$, $(2, \pi)$, $(-1, \pi/2)$
      \item Compare all three methods in terms of:
      \begin{itemize}
        \item Convergence speed
        \item Final solutions found
        \item Robustness to different starting points
      \end{itemize}
    \end{enumerate}
  \end{block}
\end{frame}


\end{document}


