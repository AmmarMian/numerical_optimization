\documentclass[aspectratio=1610]{beamer}
\usepackage[utf8]{inputenc}
% \usepackage[default,oldstyle,scale=0.95]{opensans}
% \usepackage[T1]{fontenc}
\usepackage{sansmathfonts}
\usepackage[T1]{fontenc}

\usepackage[natbib=true,style=numeric,sorting=none]{biblatex}
\addbibresource{references.bib}

\usepackage{tikz}
\usepackage{tikz-3dplot}

\usepackage{pgf-pie}  
\usepackage{caption}
\usepackage{subcaption}
\usepackage{algorithm}
\usepackage{algpseudocode}
\usepackage{amsmath,amssymb,amsfonts}
\usepackage{svg}

\usetikzlibrary{arrows.meta}
\usetikzlibrary{tikzmark}
\usetikzlibrary{math}
\usetikzlibrary{backgrounds,calc,positioning}
\usepackage{qrcode}
\usepackage{siunitx}
\usetikzlibrary{patterns,positioning,arrows.meta,decorations.pathreplacing}

\usepackage{pgfplots}
\usepackage{algorithm, algpseudocode}



\usetheme{main}

\captionsetup{font=scriptsize,labelfont=scriptsize}


\title[Numerical optimization]{Numerical optimization : theory and applications}

\date[]{}
\author[AM]{\textbf{Ammar Mian} \\ \footnotesize Associate professor, LISTIC, Université Savoie Mont Blanc}

\newcommand{\red}[1]{\textcolor{red}{#1}}
% \newcommand{\alert}[1]{{\textbf{\alert{#1}}}}


%%%%%% NEW MATH DEFINITIONS %%%%%

\usepackage{amsmath,amsfonts,bm}

% Mark sections of captions for referring to divisions of figures
\newcommand{\figleft}{{\em (Left)}}
\newcommand{\figcenter}{{\em (Center)}}
\newcommand{\figright}{{\em (Right)}}
\newcommand{\figtop}{{\em (Top)}}
\newcommand{\figbottom}{{\em (Bottom)}}
\newcommand{\captiona}{{\em (a)}}
\newcommand{\captionb}{{\em (b)}}
\newcommand{\captionc}{{\em (c)}}
\newcommand{\captiond}{{\em (d)}}

% Highlight a newly defined term
\newcommand{\newterm}[1]{{\bf #1}}


% Figure reference, lower-case.
\def\figref#1{figure~\ref{#1}}
% Figure reference, capital. For start of sentence
\def\Figref#1{Figure~\ref{#1}}
\def\twofigref#1#2{figures \ref{#1} and \ref{#2}}
\def\quadfigref#1#2#3#4{figures \ref{#1}, \ref{#2}, \ref{#3} and \ref{#4}}
% Section reference, lower-case.
\def\secref#1{section~\ref{#1}}
% Section reference, capital.
\def\Secref#1{Section~\ref{#1}}
% Reference to two sections.
\def\twosecrefs#1#2{sections \ref{#1} and \ref{#2}}
% Reference to three sections.
\def\secrefs#1#2#3{sections \ref{#1}, \ref{#2} and \ref{#3}}
% Reference to an equation, lower-case.
% \def\eqref#1{equation~\ref{#1}}
% Reference to an equation, upper case
\def\Eqref#1{Equation~\ref{#1}}
% A raw reference to an equation---avoid using if possible
\def\plaineqref#1{\ref{#1}}
% Reference to a chapter, lower-case.
\def\chapref#1{chapter~\ref{#1}}
% Reference to an equation, upper case.
\def\Chapref#1{Chapter~\ref{#1}}
% Reference to a range of chapters
\def\rangechapref#1#2{chapters\ref{#1}--\ref{#2}}
% Reference to an algorithm, lower-case.
\def\algref#1{algorithm~\ref{#1}}
% Reference to an algorithm, upper case.
\def\Algref#1{Algorithm~\ref{#1}}
\def\twoalgref#1#2{algorithms \ref{#1} and \ref{#2}}
\def\Twoalgref#1#2{Algorithms \ref{#1} and \ref{#2}}
% Reference to a part, lower case
\def\partref#1{part~\ref{#1}}
% Reference to a part, upper case
\def\Partref#1{Part~\ref{#1}}
\def\twopartref#1#2{parts \ref{#1} and \ref{#2}}

\def\ceil#1{\lceil #1 \rceil}
\def\floor#1{\lfloor #1 \rfloor}
\def\1{\bm{1}}
\newcommand{\train}{\mathcal{D}}
\newcommand{\valid}{\mathcal{D_{\mathrm{valid}}}}
\newcommand{\test}{\mathcal{D_{\mathrm{test}}}}

\def\eps{{\epsilon}}


% Random variables
\def\reta{{\textnormal{$\eta$}}}
\def\ra{{\textnormal{a}}}
\def\rb{{\textnormal{b}}}
\def\rc{{\textnormal{c}}}
\def\rd{{\textnormal{d}}}
\def\re{{\textnormal{e}}}
\def\rf{{\textnormal{f}}}
\def\rg{{\textnormal{g}}}
\def\rh{{\textnormal{h}}}
\def\ri{{\textnormal{i}}}
\def\rj{{\textnormal{j}}}
\def\rk{{\textnormal{k}}}
\def\rl{{\textnormal{l}}}
% rm is already a command, just don't name any random variables m
\def\rn{{\textnormal{n}}}
\def\ro{{\textnormal{o}}}
\def\rp{{\textnormal{p}}}
\def\rq{{\textnormal{q}}}
\def\rr{{\textnormal{r}}}
\def\rs{{\textnormal{s}}}
\def\rt{{\textnormal{t}}}
\def\ru{{\textnormal{u}}}
\def\rv{{\textnormal{v}}}
\def\rw{{\textnormal{w}}}
\def\rx{{\textnormal{x}}}
\def\ry{{\textnormal{y}}}
\def\rz{{\textnormal{z}}}

% Random vectors
\def\rvepsilon{{\mathbf{\epsilon}}}
\def\rvtheta{{\mathbf{\theta}}}
\def\rva{{\mathbf{a}}}
\def\rvb{{\mathbf{b}}}
\def\rvc{{\mathbf{c}}}
\def\rvd{{\mathbf{d}}}
\def\rve{{\mathbf{e}}}
\def\rvf{{\mathbf{f}}}
\def\rvg{{\mathbf{g}}}
\def\rvh{{\mathbf{h}}}
\def\rvu{{\mathbf{i}}}
\def\rvj{{\mathbf{j}}}
\def\rvk{{\mathbf{k}}}
\def\rvl{{\mathbf{l}}}
\def\rvm{{\mathbf{m}}}
\def\rvn{{\mathbf{n}}}
\def\rvo{{\mathbf{o}}}
\def\rvp{{\mathbf{p}}}
\def\rvq{{\mathbf{q}}}
\def\rvr{{\mathbf{r}}}
\def\rvs{{\mathbf{s}}}
\def\rvt{{\mathbf{t}}}
\def\rvu{{\mathbf{u}}}
\def\rvv{{\mathbf{v}}}
\def\rvw{{\mathbf{w}}}
\def\rvx{{\mathbf{x}}}
\def\rvy{{\mathbf{y}}}
\def\rvz{{\mathbf{z}}}

% Elements of random vectors
\def\erva{{\textnormal{a}}}
\def\ervb{{\textnormal{b}}}
\def\ervc{{\textnormal{c}}}
\def\ervd{{\textnormal{d}}}
\def\erve{{\textnormal{e}}}
\def\ervf{{\textnormal{f}}}
\def\ervg{{\textnormal{g}}}
\def\ervh{{\textnormal{h}}}
\def\ervi{{\textnormal{i}}}
\def\ervj{{\textnormal{j}}}
\def\ervk{{\textnormal{k}}}
\def\ervl{{\textnormal{l}}}
\def\ervm{{\textnormal{m}}}
\def\ervn{{\textnormal{n}}}
\def\ervo{{\textnormal{o}}}
\def\ervp{{\textnormal{p}}}
\def\ervq{{\textnormal{q}}}
\def\ervr{{\textnormal{r}}}
\def\ervs{{\textnormal{s}}}
\def\ervt{{\textnormal{t}}}
\def\ervu{{\textnormal{u}}}
\def\ervv{{\textnormal{v}}}
\def\ervw{{\textnormal{w}}}
\def\ervx{{\textnormal{x}}}
\def\ervy{{\textnormal{y}}}
\def\ervz{{\textnormal{z}}}

% Random matrices
\def\rmA{{\mathbf{A}}}
\def\rmB{{\mathbf{B}}}
\def\rmC{{\mathbf{C}}}
\def\rmD{{\mathbf{D}}}
\def\rmE{{\mathbf{E}}}
\def\rmF{{\mathbf{F}}}
\def\rmG{{\mathbf{G}}}
\def\rmH{{\mathbf{H}}}
\def\rmI{{\mathbf{I}}}
\def\rmJ{{\mathbf{J}}}
\def\rmK{{\mathbf{K}}}
\def\rmL{{\mathbf{L}}}
\def\rmM{{\mathbf{M}}}
\def\rmN{{\mathbf{N}}}
\def\rmO{{\mathbf{O}}}
\def\rmP{{\mathbf{P}}}
\def\rmQ{{\mathbf{Q}}}
\def\rmR{{\mathbf{R}}}
\def\rmS{{\mathbf{S}}}
\def\rmT{{\mathbf{T}}}
\def\rmU{{\mathbf{U}}}
\def\rmV{{\mathbf{V}}}
\def\rmW{{\mathbf{W}}}
\def\rmX{{\mathbf{X}}}
\def\rmY{{\mathbf{Y}}}
\def\rmZ{{\mathbf{Z}}}

% Elements of random matrices
\def\ermA{{\textnormal{A}}}
\def\ermB{{\textnormal{B}}}
\def\ermC{{\textnormal{C}}}
\def\ermD{{\textnormal{D}}}
\def\ermE{{\textnormal{E}}}
\def\ermF{{\textnormal{F}}}
\def\ermG{{\textnormal{G}}}
\def\ermH{{\textnormal{H}}}
\def\ermI{{\textnormal{I}}}
\def\ermJ{{\textnormal{J}}}
\def\ermK{{\textnormal{K}}}
\def\ermL{{\textnormal{L}}}
\def\ermM{{\textnormal{M}}}
\def\ermN{{\textnormal{N}}}
\def\ermO{{\textnormal{O}}}
\def\ermP{{\textnormal{P}}}
\def\ermQ{{\textnormal{Q}}}
\def\ermR{{\textnormal{R}}}
\def\ermS{{\textnormal{S}}}
\def\ermT{{\textnormal{T}}}
\def\ermU{{\textnormal{U}}}
\def\ermV{{\textnormal{V}}}
\def\ermW{{\textnormal{W}}}
\def\ermX{{\textnormal{X}}}
\def\ermY{{\textnormal{Y}}}
\def\ermZ{{\textnormal{Z}}}

% Vectors
\def\vzero{{\bm{0}}}
\def\vone{{\bm{1}}}
\def\vmu{{\bm{\mu}}}
\def\vtheta{{\bm{\theta}}}
\def\va{{\bm{a}}}
\def\vb{{\bm{b}}}
\def\vc{{\bm{c}}}
\def\vd{{\bm{d}}}
\def\ve{{\bm{e}}}
\def\vf{{\bm{f}}}
\def\vg{{\bm{g}}}
\def\vh{{\bm{h}}}
\def\vi{{\bm{i}}}
\def\vj{{\bm{j}}}
\def\vk{{\bm{k}}}
\def\vl{{\bm{l}}}
\def\vm{{\bm{m}}}
\def\vn{{\bm{n}}}
\def\vo{{\bm{o}}}
\def\vp{{\bm{p}}}
\def\vq{{\bm{q}}}
\def\vr{{\bm{r}}}
\def\vs{{\bm{s}}}
\def\vt{{\bm{t}}}
\def\vu{{\bm{u}}}
\def\vv{{\bm{v}}}
\def\vw{{\bm{w}}}
\def\vx{{\bm{x}}}
\def\vy{{\bm{y}}}
\def\vz{{\bm{z}}}

% Elements of vectors
\def\evalpha{{\alpha}}
\def\evbeta{{\beta}}
\def\evepsilon{{\epsilon}}
\def\evlambda{{\lambda}}
\def\evomega{{\omega}}
\def\evmu{{\mu}}
\def\evpsi{{\psi}}
\def\evsigma{{\sigma}}
\def\evtheta{{\theta}}
\def\eva{{a}}
\def\evb{{b}}
\def\evc{{c}}
\def\evd{{d}}
\def\eve{{e}}
\def\evf{{f}}
\def\evg{{g}}
\def\evh{{h}}
\def\evi{{i}}
\def\evj{{j}}
\def\evk{{k}}
\def\evl{{l}}
\def\evm{{m}}
\def\evn{{n}}
\def\evo{{o}}
\def\evp{{p}}
\def\evq{{q}}
\def\evr{{r}}
\def\evs{{s}}
\def\evt{{t}}
\def\evu{{u}}
\def\evv{{v}}
\def\evw{{w}}
\def\evx{{x}}
\def\evy{{y}}
\def\evz{{z}}

% Matrix
\def\mA{{\bm{A}}}
\def\mB{{\bm{B}}}
\def\mC{{\bm{C}}}
\def\mD{{\bm{D}}}
\def\mE{{\bm{E}}}
\def\mF{{\bm{F}}}
\def\mG{{\bm{G}}}
\def\mH{{\bm{H}}}
\def\mI{{\bm{I}}}
\def\mJ{{\bm{J}}}
\def\mK{{\bm{K}}}
\def\mL{{\bm{L}}}
\def\mM{{\bm{M}}}
\def\mN{{\bm{N}}}
\def\mO{{\bm{O}}}
\def\mP{{\bm{P}}}
\def\mQ{{\bm{Q}}}
\def\mR{{\bm{R}}}
\def\mS{{\bm{S}}}
\def\mT{{\bm{T}}}
\def\mU{{\bm{U}}}
\def\mV{{\bm{V}}}
\def\mW{{\bm{W}}}
\def\mX{{\bm{X}}}
\def\mY{{\bm{Y}}}
\def\mZ{{\bm{Z}}}
\def\mBeta{{\bm{\beta}}}
\def\mPhi{{\bm{\Phi}}}
\def\mLambda{{\bm{\Lambda}}}
\def\mSigma{{\bm{\Sigma}}}

% Tensor
\DeclareMathAlphabet{\mathsfit}{\encodingdefault}{\sfdefault}{m}{sl}
\SetMathAlphabet{\mathsfit}{bold}{\encodingdefault}{\sfdefault}{bx}{n}
\newcommand{\tens}[1]{\bm{\mathsfit{#1}}}
\def\tA{{\tens{A}}}
\def\tB{{\tens{B}}}
\def\tC{{\tens{C}}}
\def\tD{{\tens{D}}}
\def\tE{{\tens{E}}}
\def\tF{{\tens{F}}}
\def\tG{{\tens{G}}}
\def\tH{{\tens{H}}}
\def\tI{{\tens{I}}}
\def\tJ{{\tens{J}}}
\def\tK{{\tens{K}}}
\def\tL{{\tens{L}}}
\def\tM{{\tens{M}}}
\def\tN{{\tens{N}}}
\def\tO{{\tens{O}}}
\def\tP{{\tens{P}}}
\def\tQ{{\tens{Q}}}
\def\tR{{\tens{R}}}
\def\tS{{\tens{S}}}
\def\tT{{\tens{T}}}
\def\tU{{\tens{U}}}
\def\tV{{\tens{V}}}
\def\tW{{\tens{W}}}
\def\tX{{\tens{X}}}
\def\tY{{\tens{Y}}}
\def\tZ{{\tens{Z}}}


% Graph
\def\gA{{\mathcal{A}}}
\def\gB{{\mathcal{B}}}
\def\gC{{\mathcal{C}}}
\def\gD{{\mathcal{D}}}
\def\gE{{\mathcal{E}}}
\def\gF{{\mathcal{F}}}
\def\gG{{\mathcal{G}}}
\def\gH{{\mathcal{H}}}
\def\gI{{\mathcal{I}}}
\def\gJ{{\mathcal{J}}}
\def\gK{{\mathcal{K}}}
\def\gL{{\mathcal{L}}}
\def\gM{{\mathcal{M}}}
\def\gN{{\mathcal{N}}}
\def\gO{{\mathcal{O}}}
\def\gP{{\mathcal{P}}}
\def\gQ{{\mathcal{Q}}}
\def\gR{{\mathcal{R}}}
\def\gS{{\mathcal{S}}}
\def\gT{{\mathcal{T}}}
\def\gU{{\mathcal{U}}}
\def\gV{{\mathcal{V}}}
\def\gW{{\mathcal{W}}}
\def\gX{{\mathcal{X}}}
\def\gY{{\mathcal{Y}}}
\def\gZ{{\mathcal{Z}}}

% Sets
\def\sA{{\mathbb{A}}}
\def\sB{{\mathbb{B}}}
\def\sC{{\mathbb{C}}}
\def\sD{{\mathbb{D}}}
% Don't use a set called E, because this would be the same as our symbol
% for expectation.
\def\sF{{\mathbb{F}}}
\def\sG{{\mathbb{G}}}
\def\sH{{\mathbb{H}}}
\def\sI{{\mathbb{I}}}
\def\sJ{{\mathbb{J}}}
\def\sK{{\mathbb{K}}}
\def\sL{{\mathbb{L}}}
\def\sM{{\mathbb{M}}}
\def\sN{{\mathbb{N}}}
\def\sO{{\mathbb{O}}}
\def\sP{{\mathbb{P}}}
\def\sQ{{\mathbb{Q}}}
\def\sR{{\mathbb{R}}}
\def\sS{{\mathbb{S}}}
\def\sT{{\mathbb{T}}}
\def\sU{{\mathbb{U}}}
\def\sV{{\mathbb{V}}}
\def\sW{{\mathbb{W}}}
\def\sX{{\mathbb{X}}}
\def\sY{{\mathbb{Y}}}
\def\sZ{{\mathbb{Z}}}

% Entries of a matrix
\def\emLambda{{\Lambda}}
\def\emA{{A}}
\def\emB{{B}}
\def\emC{{C}}
\def\emD{{D}}
\def\emE{{E}}
\def\emF{{F}}
\def\emG{{G}}
\def\emH{{H}}
\def\emI{{I}}
\def\emJ{{J}}
\def\emK{{K}}
\def\emL{{L}}
\def\emM{{M}}
\def\emN{{N}}
\def\emO{{O}}
\def\emP{{P}}
\def\emQ{{Q}}
\def\emR{{R}}
\def\emS{{S}}
\def\emT{{T}}
\def\emU{{U}}
\def\emV{{V}}
\def\emW{{W}}
\def\emX{{X}}
\def\emY{{Y}}
\def\emZ{{Z}}
\def\emSigma{{\Sigma}}

% entries of a tensor
% Same font as tensor, without \bm wrapper
\newcommand{\etens}[1]{\mathsfit{#1}}
\def\etLambda{{\etens{\Lambda}}}
\def\etA{{\etens{A}}}
\def\etB{{\etens{B}}}
\def\etC{{\etens{C}}}
\def\etD{{\etens{D}}}
\def\etE{{\etens{E}}}
\def\etF{{\etens{F}}}
\def\etG{{\etens{G}}}
\def\etH{{\etens{H}}}
\def\etI{{\etens{I}}}
\def\etJ{{\etens{J}}}
\def\etK{{\etens{K}}}
\def\etL{{\etens{L}}}
\def\etM{{\etens{M}}}
\def\etN{{\etens{N}}}
\def\etO{{\etens{O}}}
\def\etP{{\etens{P}}}
\def\etQ{{\etens{Q}}}
\def\etR{{\etens{R}}}
\def\etS{{\etens{S}}}
\def\etT{{\etens{T}}}
\def\etU{{\etens{U}}}
\def\etV{{\etens{V}}}
\def\etW{{\etens{W}}}
\def\etX{{\etens{X}}}
\def\etY{{\etens{Y}}}
\def\etZ{{\etens{Z}}}

% The true underlying data generating distribution
\newcommand{\pdata}{p_{\rm{data}}}
% The empirical distribution defined by the training set
\newcommand{\ptrain}{\hat{p}_{\rm{data}}}
\newcommand{\Ptrain}{\hat{P}_{\rm{data}}}
% The model distribution
\newcommand{\pmodel}{p_{\rm{model}}}
\newcommand{\Pmodel}{P_{\rm{model}}}
\newcommand{\ptildemodel}{\tilde{p}_{\rm{model}}}
% Stochastic autoencoder distributions
\newcommand{\pencode}{p_{\rm{encoder}}}
\newcommand{\pdecode}{p_{\rm{decoder}}}
\newcommand{\precons}{p_{\rm{reconstruct}}}

\newcommand{\laplace}{\mathrm{Laplace}} % Laplace distribution

\newcommand{\E}{\mathbb{E}}
\newcommand{\Ls}{\mathcal{L}}
% \newcommand{\R}{\mathbb{R}}
\newcommand{\emp}{\tilde{p}}
\newcommand{\lr}{\alpha}
\newcommand{\reg}{\lambda}
\newcommand{\rect}{\mathrm{rectifier}}
\newcommand{\softmax}{\mathrm{softmax}}
\newcommand{\sigmoid}{\sigma}
\newcommand{\softplus}{\zeta}
\newcommand{\KL}{D_{\mathrm{KL}}}
\newcommand{\Var}{\mathrm{Var}}
\newcommand{\standarderror}{\mathrm{SE}}
\newcommand{\Cov}{\mathrm{Cov}}
% Wolfram Mathworld says $L^2$ is for function spaces and $\ell^2$ is for vectors
% But then they seem to use $L^2$ for vectors throughout the site, and so does
% wikipedia.
\newcommand{\normlzero}{L^0}
\newcommand{\normlone}{L^1}
\newcommand{\normltwo}{L^2}
\newcommand{\normlp}{L^p}
\newcommand{\normmax}{L^\infty}

\newcommand{\parents}{Pa} % See usage in notation.tex. Chosen to match Daphne's book.

% \DeclareMathOperator*{\argmax}{arg\,max}
% \DeclareMathOperator*{\argmin}{arg\,min}

\DeclareMathOperator{\sign}{sign}
\DeclareMathOperator{\Tr}{Tr}
\let\ab\allowbreak


\newcommand{\mat}[1]{\mathbf{#1}}
\renewcommand{\vec}[1]{\boldsymbol{#1}}
\renewcommand{\det}[1]{\lvert #1 \rvert}
\newcommand{\Esp}[2]{\mathbb{E}_{#1}\left[#2 \right]}
\renewcommand{\v}{\vec{v}}
\newcommand{\eye}[1]{\mathbf{I}_{#1}}
\newcommand{\norm}[1]{\left\| #1 \right\|}
\newcommand{\argmin}[1]{\underset{#1}{\operatorname{argmin}}}
\newcommand{\argmax}[1]{\underset{#1}{\operatorname{argmax}}}

% \newcommand{\bm}[1]{\boldsymbol{#1}}
\def\R{{\mathbb{R}}}
\def\K{{\mathbb{K}}}
\def\C{{\mathbb{C}}}
\def\N{{\mathbb{N}}}
\def\Z{{\mathbb{Z}}}
\def\d{{\rm d}}


\def\eme{{{\grave{e}me}}}

\def\m{{\mbox{m}}}
\def\Hz{{\mbox{Hz}}}
\def\Watt{{\mbox{W}}}

\DeclareMathOperator{\vecop}{vec}
\DeclareMathOperator{\unvec}{vec^{-1}}
\DeclareMathOperator{\tr}{tr}
\DeclareMathOperator{\diag}{diag}
\DeclareMathOperator{\prox}{prox}
\DeclareMathOperator{\rank}{rank}
\DeclareMathOperator{\complexj}{\mathsf{j}}
\DeclareMathOperator{\proj}{proj}
\DeclareMathOperator{\sinc}{sinc}






\AtBeginBibliography{\scriptsize}

%%%%%%%%%%%%%%%%%%%%%%%%%%%%%%%%%%%%%%%%%%%%%%%%%%%%%%%%%%%%
%                     END OF PREAMBLE
%
%%%%%%%%%%%%%%%%%%%%%%%%%%%%%%%%%%%%%%%%%%%%%%%%%%%%%%%%%%%%

\begin{document}
%%%%%%%%%%%%%%%


  
\begin{frame}[noframenumbering,plain]
\titlepage
\end{frame}
\begingroup
\setbeamercolor{background canvas}{bg=main}
\setbeamercolor{titlelike}{fg=text-light, bg=light}
\begin{frame}[noframenumbering,plain]{Outline}
    \tableofcontents[]
\end{frame}

\endgroup

\AtBeginSection[]{
        \setbeamercolor{background canvas}{bg=main} 
    \begin{frame}[plain, noframenumbering]
        \tableofcontents[currentsection]
    \end{frame}
    \setbeamercolor{background canvas}{bg=light} 
}


\AtBeginSubsection[]
{
    \setbeamercolor{background canvas}{bg=main} 
    \begin{frame}[noframenumbering, plain, label=]
        % \frametitle{Plan}  
        \tableofcontents[currentsection,currentsubsection]
    \end{frame}
    \setbeamercolor{background canvas}{bg=light} 
}



%%%%%%%%%%%%%%%%%%%%%%%%%%%%%%%%%%%%%%%%%%%%%%%%%%%%%%%%%%%%
%%%%%%%%%%%%%%%%%%%%%%%%%%%%%%%%%%%%%%%%%%%%%%%%%%%%%%%%%%%%
\begin{frame}{The Missing Piece: Understanding the Saddle Point Structure}
  \textbf{What we covered previously:} KKT conditions tell us \emph{what} the solution looks like
  
  \vspace{0.3cm}
  \textbf{What we missed:} \emph{How} to optimize the Lagrangian to find this solution
  
  \vspace{0.5cm}
  \begin{block}{Key Question}
    Given $\mathcal{L}(\mathbf{x}, \boldsymbol{\lambda}) = f(\mathbf{x}) - \sum_{i} \lambda_i c_i(\mathbf{x})$, how do we optimize over $(\mathbf{x}, \boldsymbol{\lambda})$?
  \end{block}
  
  \vspace{0.3cm}
  \textbf{The fundamental insight:} The KKT conditions emerge from a \emph{saddle point} structure where:
  \begin{itemize}
    \item We \textcolor{blue}{\textbf{minimize}} over primal variables $\mathbf{x}$
    \item We \textcolor{red}{\textbf{maximize}} over dual variables $\boldsymbol{\lambda} \geq 0$
  \end{itemize}
  
  \vspace{0.3cm}
  This opposite optimization behavior is \emph{not} arbitrary---it emerges naturally from the mathematical structure of constrained optimization.
\end{frame}

\begin{frame}{Why the Minus Sign Creates the Right Incentives}
  Consider our Lagrangian: $\mathcal{L}(\mathbf{x}, \boldsymbol{\lambda}) = f(\mathbf{x}) - \sum_{i} \lambda_i c_i(\mathbf{x})$
  
  \vspace{0.3cm}
  \textbf{What happens if we minimize over both variables?}
  
  For inequality constraint $c_i(\mathbf{x}) \geq 0$:
  \begin{itemize}
    \item When $c_i(\mathbf{x}) > 0$ (constraint satisfied with slack)
    \item Term $-\lambda_i c_i(\mathbf{x})$ becomes more negative as $\lambda_i$ increases
    \item Minimizing over $\lambda_i$ would drive $\mathcal{L} \to +\infty$ 
  \end{itemize}
  
  \vspace{0.3cm}
  \begin{alertblock}{The Resolution}
    We must \textbf{maximize} over $\lambda_i \geq 0$. When $c_i(\mathbf{x}) > 0$, maximization drives $\lambda_i \to 0$, consistent with complementarity: $\lambda_i c_i(\mathbf{x}) = 0$.
  \end{alertblock}
  
  \vspace{0.3cm}
  The minus sign in the Lagrangian creates the correct incentive structure for the dual variables to encode constraint shadow prices.
\end{frame}

\begin{frame}{The Saddle Point Property}
  \begin{theorem}[Saddle Point Characterization]
    $(\mathbf{x}^{\star}, \boldsymbol{\lambda}^{\star})$ solves the constrained optimization problem if and only if it is a saddle point of the Lagrangian:
    $$\mathcal{L}(\mathbf{x}^{\star}, \boldsymbol{\lambda}) \leq \mathcal{L}(\mathbf{x}^{\star}, \boldsymbol{\lambda}^{\star}) \leq \mathcal{L}(\mathbf{x}, \boldsymbol{\lambda}^{\star})$$
    for all feasible $\mathbf{x}$ and all $\boldsymbol{\lambda} \geq 0$.
  \end{theorem}
  
  \vspace{0.3cm}
  \textbf{Interpretation:}
  \begin{itemize}
    \item \textcolor{blue}{Left inequality}: $\mathcal{L}(\mathbf{x}^{\star}, \boldsymbol{\lambda})$ is \emph{maximized} over $\boldsymbol{\lambda}$ at $\boldsymbol{\lambda}^{\star}$
    \item \textcolor{red}{Right inequality}: $\mathcal{L}(\mathbf{x}, \boldsymbol{\lambda}^{\star})$ is \emph{minimized} over $\mathbf{x}$ at $\mathbf{x}^{\star}$
  \end{itemize}
  
  \vspace{0.3cm}
  \begin{block}{Economic Insight}
    Dual variables $\boldsymbol{\lambda}^{\star}$ represent \textbf{shadow prices}---the marginal value of relaxing constraints. Maximization over $\boldsymbol{\lambda}$ finds the economically meaningful constraint valuations.
  \end{block}
\end{frame}

\begin{frame}{Illustrative Example: The Saddle Point in Action}
  \textbf{Problem:} $\min f(x) = -(x-3)^2$ subject to $x \geq 1$
  
  \textbf{Lagrangian:} $\mathcal{L}(x,\lambda) = -(x-3)^2 - \lambda(x-1)$
  
  \vspace{0.3cm}
  \textbf{The conflict:} Objective wants $x \to -\infty$, constraint forces $x^{\star} = 1$
  
  \vspace{0.3cm}
  \textbf{Saddle point analysis:}
  \begin{align}
    \frac{\partial \mathcal{L}}{\partial x} &= -2(x-3) - \lambda = 0 \quad \text{(Stationarity)}\\
    \text{At } x^{\star} = 1: \quad &-2(1-3) - \lambda = 0 \Rightarrow \lambda^{\star} = 4
  \end{align}
  
  \vspace{0.3cm}
  \textbf{Verification of saddle property:}
  \begin{itemize}
    \item Fix $\lambda = 4$: $\mathcal{L}(x,4) = -(x-3)^2 - 4(x-1)$ has unique minimum at $x = 1$
    \item Fix $x = 1$: $\mathcal{L}(1,\lambda) = -4$ (constant, satisfying max condition)
  \end{itemize}
  
  \textbf{Shadow price:} $\lambda^{\star} = 4$ means relaxing $x \geq 1$ to $x \geq 1-\epsilon$ improves objective by $\approx 4\epsilon$.
\end{frame}

\begin{frame}{From Theory to Algorithm: Projected Gradient Method}
  The saddle point structure naturally suggests an \textbf{alternating optimization} scheme:
  
  \vspace{0.5cm}
  \begin{block}{Projected Gradient Algorithm}
    \textbf{Initialize:} $\mathbf{x}^0, \boldsymbol{\lambda}^0 \geq 0$
    
    \textbf{For} $k = 0, 1, 2, \ldots$ \textbf{until convergence:}
    \begin{align}
      \mathbf{x}^{k+1} &= \mathbf{x}^k - \alpha_k \nabla_{\mathbf{x}} \mathcal{L}(\mathbf{x}^k, \boldsymbol{\lambda}^k) \tag{Primal descent}\\
      \boldsymbol{\lambda}^{k+1} &= \max(0, \boldsymbol{\lambda}^k + \beta_k \nabla_{\boldsymbol{\lambda}} \mathcal{L}(\mathbf{x}^{k+1}, \boldsymbol{\lambda}^k)) \tag{Dual ascent}
    \end{align}
  \end{block}
  
  \vspace{0.3cm}
  \textbf{Key components:}
  \begin{itemize}
    \item \textcolor{blue}{\textbf{Primal step}}: Gradient descent on $\mathcal{L}$ with respect to $\mathbf{x}$
    \item \textcolor{red}{\textbf{Dual step}}: Projected gradient ascent on $\mathcal{L}$ with respect to $\boldsymbol{\lambda}$
    \item \textbf{Projection}: $\max(0, \cdot)$ ensures dual feasibility $\boldsymbol{\lambda} \geq 0$
  \end{itemize}
\end{frame}

\begin{frame}{Understanding the Gradient Components}
  For our general Lagrangian $\mathcal{L}(\mathbf{x}, \boldsymbol{\lambda}) = f(\mathbf{x}) - \sum_{i} \lambda_i c_i(\mathbf{x})$:
  
  \vspace{0.3cm}
  \textbf{Primal gradient:}
  $$\nabla_{\mathbf{x}} \mathcal{L}(\mathbf{x}, \boldsymbol{\lambda}) = \nabla f(\mathbf{x}) - \sum_{i} \lambda_i \nabla c_i(\mathbf{x})$$
  
  \textbf{Dual gradient:}
  $$\frac{\partial \mathcal{L}}{\partial \lambda_i} = -c_i(\mathbf{x})$$
  
  \vspace{0.3cm}
  \begin{block}{Algorithm Updates}
    \begin{align}
      \mathbf{x}^{k+1} &= \mathbf{x}^k - \alpha_k \left(\nabla f(\mathbf{x}^k) - \sum_{i} \lambda_i^k \nabla c_i(\mathbf{x}^k)\right)\\
      \lambda_i^{k+1} &= \max(0, \lambda_i^k + \beta_k c_i(\mathbf{x}^{k+1})) \quad \forall i
    \end{align}
  \end{block}
  
  \textbf{Intuition:} Dual variables increase when constraints are violated ($c_i < 0$) and decrease when constraints have slack ($c_i > 0$), naturally driving toward complementarity.
\end{frame}

\begin{frame}{Algorithm Implementation for Our Exercise}
  \textbf{Recall our problem:}
  \begin{align}
    \text{minimize} \quad & f(x,y) = (x-2)^2 + (y-2)^2 \\
    \text{subject to:} \quad & g(x,y) = x + y - 2 = 0 \\
    & h_1(x,y) = -x \leq 0, \quad h_2(x,y) = -y \leq 0
  \end{align}
  
  \textbf{Lagrangian:}
  $$\mathcal{L}(x,y,\lambda,\mu_1,\mu_2) = (x-2)^2 + (y-2)^2 - \lambda(x + y - 2) - \mu_1(-x) - \mu_2(-y)$$
  
  \textbf{Gradients:}
  \begin{align}
    \frac{\partial \mathcal{L}}{\partial x} &= 2(x-2) - \lambda + \mu_1\\
    \frac{\partial \mathcal{L}}{\partial y} &= 2(y-2) - \lambda + \mu_2\\
    \frac{\partial \mathcal{L}}{\partial \lambda} &= -(x + y - 2)\\
    \frac{\partial \mathcal{L}}{\partial \mu_1} &= x, \quad \frac{\partial \mathcal{L}}{\partial \mu_2} = y
  \end{align}
\end{frame}

\begin{frame}{Projected Gradient Steps for Our Exercise}
  \textbf{Algorithm updates:}
  \begin{align}
    x^{k+1} &= x^k - \alpha(2(x^k-2) - \lambda^k + \mu_1^k)\\
    y^{k+1} &= y^k - \alpha(2(y^k-2) - \lambda^k + \mu_2^k)\\
    \lambda^{k+1} &= \lambda^k + \beta(x^{k+1} + y^{k+1} - 2)\\
    \mu_1^{k+1} &= \max(0, \mu_1^k - \beta x^{k+1})\\
    \mu_2^{k+1} &= \max(0, \mu_2^k - \beta y^{k+1})
  \end{align}
  
  \vspace{0.3cm}
  \textbf{Expected convergence:} $(x^{\star}, y^{\star}) = (1, 1)$ with $\lambda^{\star} = 2$, $\mu_1^{\star} = \mu_2^{\star} = 0$
  
  \vspace{0.3cm}
  \begin{alertblock}{Key Insight}
    The inequality constraints $x \geq 0, y \geq 0$ are \textbf{inactive} at the solution because the optimal point $(1,1)$ lies in the interior of the feasible region. Therefore $\mu_1^{\star} = \mu_2^{\star} = 0$ by complementarity.
  \end{alertblock}
\end{frame}


\begin{frame}{Corrected Implementation and Key Takeaways}
  \textbf{Implementation insight:} The projected gradient method will automatically handle the constraint activity determination through the projection steps.
  
  \vspace{0.3cm}
  \textbf{Algorithm behavior:}
  \begin{itemize}
    \item Algorithm starts with some initial guess
    \item Primal variables evolve toward $(1,1)$ due to objective function pull
    \item Dual variables for inactive constraints get projected to zero
    \item Equality constraint multiplier adjusts to maintain stationarity
  \end{itemize}
  
  \vspace{0.5cm}
  \begin{block}{Main Learning Objectives}
    \textbf{1.} Saddle point structure emerges from constraint-objective conflicts
    
    \textbf{2.} Opposite optimization directions (min over $\mathbf{x}$, max over $\boldsymbol{\lambda}$) are mathematically necessary
    
    \textbf{3.} Projected gradient algorithm implements this structure computationally
    
    \textbf{4.} Shadow prices have economic meaning as constraint relaxation values
  \end{block}
\end{frame}
\end{document}


